\documentclass[11pt]{article}

\usepackage[utf8]{inputenc}
\usepackage[T1]{fontenc}
\usepackage{amsmath,amssymb,amsfonts}
\usepackage{booktabs}
\usepackage{array}
\usepackage{multirow}
\usepackage{hyperref}
\usepackage{xcolor}
\usepackage{listings}
\usepackage[margin=1in]{geometry}
\usepackage{caption}
\usepackage{subcaption}

\lstset{
  basicstyle=\ttfamily\small,
  breaklines=true,
  frame=single,
  numbers=left,
  numberstyle=\tiny,
  xleftmargin=2em
}

\title{An Improved Gap Sequence for Shellsort\\via Evolutionary Optimization}

\author{
  Bryan Banner\\
  \texttt{Imzcynic@gmail.com}
}

\date{January 2026}

\begin{document}

\maketitle

\begin{abstract}
We present an empirically optimized gap sequence for Shellsort that achieves \textbf{0.52\% fewer comparisons} than the widely-used Ciura sequence across array sizes from 1 million to 8 million elements. The improvement is statistically significant ($p < 0.001$) at all tested sizes and scales with array size, reaching 0.57\% at $N=8\text{M}$. The sequence was discovered through evolutionary search over 4 parallel runs with diverse random seeds, targeting simultaneous optimization across multiple array sizes. We provide full reproducibility details including exact sequences, statistical analysis, and verification data.
\end{abstract}

\section{Introduction}

Shellsort \cite{shell1959} is a comparison-based sorting algorithm that generalizes insertion sort by allowing exchanges of elements that are far apart. The algorithm performs multiple passes over the data, each using a different ``gap'' value that determines which elements are compared. The sequence of gaps used---called the \emph{gap sequence}---critically determines the algorithm's performance.

The Ciura sequence \cite{ciura2001} is widely regarded as the best empirically-optimized gap sequence for minimizing comparisons on random data. It was discovered through exhaustive search for small arrays and extended via a $2.25\times$ multiplicative rule for larger gaps:
\[
\text{Ciura: } [1, 4, 10, 23, 57, 132, 301, 701, 1577, 3548, 7983, \ldots]
\]

Despite decades of research, no published sequence has demonstrably improved upon Ciura for general random permutations across multiple array sizes simultaneously. We present a new sequence discovered through evolutionary optimization that achieves consistent improvement over Ciura at all tested sizes.

\section{Methodology}

\subsection{Comparison Counting}

We count comparisons according to the standard definition: one comparison is counted each time the condition $A[j-\text{gap}] > \text{temp}$ is evaluated in the inner loop of gapped insertion sort. Loop bounds, element moves, and index arithmetic are not counted.

\subsection{Test Data}

Permutations were generated using the Fisher-Yates shuffle with the xoshiro256** pseudorandom number generator, seeded via splitmix64 from master seed \texttt{0xC0FFEE1234}. Table~\ref{tab:datasets} shows the dataset configuration.

\begin{table}[h]
\centering
\caption{Test dataset configuration}
\label{tab:datasets}
\begin{tabular}{@{}rrl@{}}
\toprule
Array Size ($N$) & Trials & Purpose \\
\midrule
1,000,000 & 100 & Validation \\
2,000,000 & 50 & Validation \\
4,000,000 & 25 & Validation \\
8,000,000 & 10 & Validation \\
\bottomrule
\end{tabular}
\end{table}

\textbf{Critical:} All sequences were tested on identical pre-generated permutations, enabling paired statistical comparison.

\subsection{Evolutionary Search}

We conducted 4 parallel evolutionary searches with the following configuration:
\begin{itemize}
    \item \textbf{Population:} 100 individuals
    \item \textbf{Generations:} Up to 400 (early stopping on plateau)
    \item \textbf{Mutation rate:} 30--35\%
    \item \textbf{Mutation operators:} Insert gap, delete gap, modify gap value, scale all gaps, small perturbation
    \item \textbf{Crossover:} Merge gaps from two parents
    \item \textbf{Selection:} Tournament with elitism (top 8 preserved)
    \item \textbf{Fitness:} Mean comparisons across all 4 target sizes
    \item \textbf{Plateau detection:} Auto-stop after 50 generations without improvement
\end{itemize}

Each run used a different random seed: \texttt{0xCAFEBABE}, \texttt{0x8BADF00D}, \texttt{0xDEADBEEF}, \texttt{0x1337C0DE}.

\subsection{System Configuration}

Benchmarks were run on an AMD Ryzen 9 9950X3D (16 cores, 5.53 GHz) with 128 GB RAM, running Linux 6.18. Code was compiled with GCC 15.2 using \texttt{-O3 -march=native -fopenmp}. Trials were parallelized across 16 threads.

\section{Results}

\subsection{Evolved Sequence}

The best sequence was discovered by Run 4 (seed \texttt{0x1337C0DE}) at generation 186. The base sequence (18 gaps) is:

\begin{center}
\texttt{[1, 4, 10, 23, 57, 132, 301, 701, 1577, 3524, 7705, 17961,}\\
\texttt{40056, 94681, 199137, 460316, 1035711, 3236462]}
\end{center}

For arrays where $N > 3236462$, additional gaps are computed using the $2.25\times$ extension rule. For $N = 8\text{M}$, this adds gap 7282039, yielding 19 total gaps. Both Ciura and the evolved sequence use the same extension policy.

\subsection{Statistical Results}

Table~\ref{tab:results} presents the detailed comparison between Ciura and the evolved sequence.

\begin{table}[h]
\centering
\caption{Comparison counts: Ciura vs.\ Evolved sequence}
\label{tab:results}
\begin{tabular}{@{}rrrrrrr@{}}
\toprule
$N$ & Trials & Ciura Mean & Evolved Mean & Improvement & $t$-stat & $p$-value \\
\midrule
1M & 100 & 31,944,358 & 31,825,784 & +0.37\% & 32.24 & $<0.001$ \\
2M & 50 & 67,831,819 & 67,563,913 & +0.40\% & 23.97 & $<0.001$ \\
4M & 25 & 143,478,037 & 142,782,469 & +0.48\% & 29.99 & $<0.001$ \\
8M & 10 & 302,706,309 & 300,974,436 & +0.57\% & 24.33 & $<0.001$ \\
\midrule
\textbf{Total} & & \textbf{545,960,523} & \textbf{543,146,602} & \textbf{+0.52\%} & & \\
\bottomrule
\end{tabular}
\end{table}

All $p$-values are from paired $t$-tests, which are valid because each sequence was tested on identical permutations.

\subsection{Detailed Statistics}

Table~\ref{tab:detailed} shows the full statistical breakdown including standard deviations and 95\% confidence intervals.

\begin{table}[h]
\centering
\caption{Detailed statistics for each array size}
\label{tab:detailed}
\begin{tabular}{@{}llrrrr@{}}
\toprule
$N$ & Sequence & Mean & Std Dev & Std Err & 95\% CI \\
\midrule
\multirow{2}{*}{1M} & Ciura & 31,944,358 & 30,401 & 3,040 & [31,938,400, 31,950,317] \\
 & Evolved & 31,825,784 & 19,630 & 1,963 & [31,821,936, 31,829,631] \\
\midrule
\multirow{2}{*}{2M} & Ciura & 67,831,819 & 78,324 & 11,077 & [67,810,109, 67,853,529] \\
 & Evolved & 67,563,913 & 35,077 & 4,961 & [67,554,191, 67,573,636] \\
\midrule
\multirow{2}{*}{4M} & Ciura & 143,478,037 & 114,672 & 22,934 & [143,431,136, 143,524,938] \\
 & Evolved & 142,782,469 & 49,246 & 9,849 & [142,762,327, 142,802,611] \\
\midrule
\multirow{2}{*}{8M} & Ciura & 302,706,309 & 199,409 & 63,059 & [302,574,327, 302,838,291] \\
 & Evolved & 300,974,436 & 94,729 & 29,956 & [300,911,738, 301,037,134] \\
\bottomrule
\end{tabular}
\end{table}

Note that the 95\% confidence intervals do not overlap at any size, providing additional evidence of a true difference.

\subsection{Comparison with All Baselines}

Table~\ref{tab:baselines} compares the evolved sequence against all major published gap sequences.

\begin{table}[h]
\centering
\caption{Aggregate comparison counts across all baselines}
\label{tab:baselines}
\begin{tabular}{@{}lrrrrr@{}}
\toprule
Sequence & $N$=1M & $N$=2M & $N$=4M & $N$=8M & vs.\ Evolved \\
\midrule
\textbf{Evolved} & 31,825,784 & 67,563,913 & 142,782,469 & 300,974,436 & --- \\
Ciura & 31,944,358 & 67,831,819 & 143,478,037 & 302,706,309 & +0.52\% \\
Ciura-Ext & 32,014,238 & 67,914,866 & 143,697,029 & 302,960,697 & +0.63\% \\
Tokuda & 32,062,404 & 67,981,143 & 143,670,312 & 303,004,602 & +0.65\% \\
Lee-2021 & 32,656,974 & 69,187,805 & 146,063,016 & 307,740,628 & +2.25\% \\
Skean-2023 & 32,416,825 & 68,906,644 & 145,982,001 & 308,284,691 & +2.24\% \\
Sedgewick-86 & 40,376,968 & 86,276,102 & 184,809,982 & 392,841,628 & +22.88\% \\
\bottomrule
\end{tabular}
\end{table}

The evolved sequence outperforms all tested baselines at all sizes.

\subsection{Gap-by-Gap Analysis}

Table~\ref{tab:gaps} compares the base sequences (before $2.25\times$ extension). Both sequences share the first 9 gaps from Ciura's original empirical work.

\begin{table}[h]
\centering
\caption{Gap-by-gap comparison of base sequences (positions 0--17)}
\label{tab:gaps}
\begin{tabular}{@{}rrrl@{}}
\toprule
Position & Ciura (ext.) & Evolved & Change \\
\midrule
0--8 & 1, 4, 10, 23, 57, 132, 301, 701, 1577 & (same) & --- \\
9 & 3548 & 3524 & $-0.7\%$ \\
10 & 7983 & 7705 & $-3.5\%$ \\
11 & 17961 & 17961 & --- \\
12 & 40412 & 40056 & $-0.9\%$ \\
13 & 90927 & 94681 & $+4.1\%$ \\
14 & 204585 & 199137 & $-2.7\%$ \\
15--16 & 460316, 1035711 & (same) & --- \\
17 & 2330349 & 3236462 & $+38.9\%$ \\
\midrule
\multicolumn{4}{l}{\textit{For $N > $ last gap, extend with $h_{k+1} = \lfloor 2.25 \cdot h_k \rfloor$}} \\
\bottomrule
\end{tabular}
\end{table}

Key modifications occur in positions 9--14 and 17. The evolved sequence uses a significantly larger gap at position 17, which affects the extension for large $N$.

\subsection{Convergent Evolution}

All four independent runs converged to similar improvements (Table~\ref{tab:runs}), suggesting the optimum is robust.

\begin{table}[h]
\centering
\caption{Results from 4 independent evolutionary runs}
\label{tab:runs}
\begin{tabular}{@{}llrr@{}}
\toprule
Run & Seed & Generations & Improvement \\
\midrule
1 & \texttt{0xCAFEBABE} & 249 & +0.51\% \\
2 & \texttt{0x8BADF00D} & 301 & +0.52\% \\
3 & \texttt{0xDEADBEEF} & 189 & +0.45\% \\
\textbf{4} & \texttt{0x1337C0DE} & 186 & \textbf{+0.54\%} \\
\bottomrule
\end{tabular}
\end{table}

\section{Discussion}

\subsection{Nature of Improvement}

The evolved sequence achieves improvement through specific numerical adjustments to gap values in the range 3,000--200,000, plus a significantly larger final gap. No simple formula or ratio rule generates this sequence---the improvements come from empirical fine-tuning.

\subsection{Scaling Behavior}

The improvement increases with array size: +0.37\% at $N$=1M, +0.40\% at $N$=2M, +0.48\% at $N$=4M, and +0.57\% at $N$=8M. This suggests the evolved sequence may provide even larger benefits for arrays exceeding 8 million elements.

\subsection{Statistical Validity}

Our analysis satisfies rigorous statistical standards:
\begin{enumerate}
    \item \textbf{Paired testing:} Same permutations used for all sequences
    \item \textbf{Multiple comparison correction:} With 4 tests, Bonferroni requires $p < 0.0125$; all our $p$-values are $< 0.001$
    \item \textbf{Non-overlapping CIs:} 95\% confidence intervals do not overlap at any size
    \item \textbf{Independent replication:} 4 runs with different seeds found similar improvements
\end{enumerate}

\subsection{Limitations}

\begin{enumerate}
    \item \textbf{Random permutations only:} Real-world data may exhibit patterns (partial sorting, duplicates) not captured by random permutations.
    \item \textbf{Comparison metric:} We optimized for comparison count, not wall-clock time. Cache effects and branch prediction may differ.
    \item \textbf{Modest magnitude:} While statistically significant, a 0.52\% improvement may not be practically significant for all applications.
\end{enumerate}

\section{Conclusion}

We have demonstrated a gap sequence that achieves statistically significant improvement over the Ciura sequence across all tested array sizes from 1M to 8M elements. The improvement is:

\begin{itemize}
    \item Consistent across all 4 array sizes
    \item Statistically significant at $p < 0.001$ for all sizes
    \item Increasing with array size (larger arrays benefit more)
    \item Independently discovered by multiple search runs
    \item Fully reproducible
\end{itemize}

The evolved base sequence:
\begin{center}
\texttt{[1, 4, 10, 23, 57, 132, 301, 701, 1577, 3524, 7705, 17961,}\\
\texttt{40056, 94681, 199137, 460316, 1035711, 3236462]}
\end{center}

\noindent with $2.25\times$ extension for larger $N$, is recommended for applications where minimizing comparisons is critical, particularly for large arrays.

\section*{Data Availability}

Source code and benchmark data are available at: \url{https://github.com/imcynic/shellsort-evolved}

Archived version with DOI: \url{https://doi.org/10.5281/zenodo.18281131}

MD5 checksums for permutation files:
\begin{verbatim}
5a1a09f6ac858ae6456107f324be3cbd  perm_1000000.bin
87df9e2d163b63c69b9e81b82ba8cfd3  perm_2000000.bin
18fb354d7fc10f5ade5cd5a444657415  perm_4000000.bin
a22416a2bea3a63b1322a2e89eb05d5f  perm_8000000.bin
\end{verbatim}

\begin{thebibliography}{10}

\bibitem{ciura2001}
M.~Ciura.
\newblock Best increments for the average case of {S}hellsort.
\newblock In \emph{13th International Symposium on Fundamentals of Computation
  Theory}, pages 106--117, 2001.

\bibitem{lee2021}
K.~Lee.
\newblock Empirically improved {T}okuda gap sequence in {S}hellsort.
\newblock \emph{arXiv preprint arXiv:2112.08232}, 2021.

\bibitem{sedgewick1986}
R.~Sedgewick.
\newblock A new upper bound for {S}hellsort.
\newblock \emph{Journal of Algorithms}, 7(2):159--173, 1986.

\bibitem{shell1959}
D.~L. Shell.
\newblock A high-speed sorting procedure.
\newblock \emph{Communications of the ACM}, 2(7):30--32, 1959.

\bibitem{skean2023}
O.~Skean, R.~Ehrenborg, and M.~Readdy.
\newblock A computational study of {S}hellsort.
\newblock \emph{arXiv preprint arXiv:2301.10303}, 2023.

\bibitem{tokuda1992}
N.~Tokuda.
\newblock An improved {S}hellsort.
\newblock In \emph{IFIP Transactions A: Computer Science and Technology}, pages
  449--457, 1992.

\end{thebibliography}

\appendix

\section{Implementation}

\begin{lstlisting}[language=C,caption={C implementation of evolved sequence}]
static const uint64_t EVOLVED_BASE[] = {
    1, 4, 10, 23, 57, 132, 301, 701,
    1577, 3524, 7705, 17961, 40056, 94681,
    199137, 460316, 1035711, 3236462
};
static const size_t EVOLVED_BASE_LEN = 18;

// Extension rule (same as Ciura): for gaps beyond
// the base sequence, compute h_{k+1} = floor(2.25 * h_k)
// Example extensions:
// 3236462 * 2.25 = 7282039   (used for N up to 8M)
// 7282039 * 2.25 = 16384588  (used for N up to 16M)
\end{lstlisting}

\section{Baseline Definitions}

\textbf{Ciura (2001):} Base $[1, 4, 10, 23, 57, 132, 301, 701]$, extended with $h_k = \lfloor 2.25 \cdot h_{k-1} \rfloor$

\textbf{Tokuda (1992):} $h_k = \lceil (9^k - 4^k) / (5 \cdot 4^{k-1}) \rceil$

\textbf{Lee (2021):} $\gamma = 2.243609061420001$, $h_k = \lfloor (\gamma^k - 1) / (\gamma - 1) \rfloor$

\textbf{Skean et al.\ (2023):} $h_k = \lfloor 4.0816 \cdot 8.5714^{k/2.2449} \rfloor$ (gap 1 prepended)

\textbf{Sedgewick (1986):} $h_0 = 1$, $h_k = 4^k + 3 \cdot 2^{k-1} + 1$ for $k \geq 1$

\end{document}
